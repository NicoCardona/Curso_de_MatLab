\documentclass{article}
\usepackage[utf8]{inputenc}
\usepackage[spanish]{babel}
\usepackage{xcolor}
\usepackage{wrapfig}
\usepackage{amsmath}
\usepackage{amsfonts}
\usepackage{amssymb}
\usepackage{graphicx}
\usepackage{hyperref}
\usepackage{lipsum}
\usepackage{multicol}


\date{\today}
\author{Nicolas Cardona Ramirez}


\begin{document}
	
	\title{\textcolor{violet}{Curso de MatLab}}
	\maketitle
	
	\section{Episodio 10: Formato de Salida}
	
	\subsection{\textcolor{blue}{Notación Científica}}
	
	Se utiliza para escribir número con un valor muy grande o muy pequeño. En MatLab se utiliza:
	
	\begin{itemize}
		\item  6.022e23
	\end{itemize}

	\subsection{\textcolor{blue}{Cambiar formato de Números}}
	
	\begin{itemize}
	
	\item Para utilizar 14 cifras decimales se utiliza \textcolor{green}{format long}
	
	\item Para utiliza 2 cifras decimales se utiliza
	\textcolor{green}{format bank} y se aproxima la última cifra decimal
	
	\item El formato por defecto con 4 cifras decimales es \textcolor{green}{format short}
	\item El \textcolor{green}{format +} despliega solo los signos positivos y negativos dentro de una matriz
	
	\item El \textcolor{green}{format rat} despliega los números como números racionales, es decir, como fracciones
	
	\item Para cambiar la forma de la notación científica se utiliza los comandos anteriores más la letra e; \textcolor{green}{format short e}
	
	\end{itemize}
	
	
	\section{Episodio 11: Funciones en MatLab}




	\subsection{\textcolor{blue}{Funciones Internas}}
	
	Las funciones internas más utilizadas son:
	
	\begin{itemize}
		\item Para usar raíz cuadrada se utiliza \textcolor{green}{sqrt()} y puede ser una matriz o un vector
		\item La función \textcolor{green}{rem(a,b)} calcula el residuo de dos números
		\item La función \textcolor{green}{[x,y] = size(d)} calcula el tamaño de una matriz o de un vector y tiene dos parámetros de salida donde \colorbox{orange}{x} es el número de filas y \colorbox{orange}{y}
		
	\end{itemize}
	
	\subsection{\textcolor{blue}{Funciones anidadas}}

	Para realizar una función anidada se utiliza \textcolor{green}{sqrt(sin(x))}

	\section{Episodio 12: Funciones Matemáticas}
	
	Las funciones matemáticas esenciales son:
	
	\begin{itemize}
		\item \textcolor{red}{abs()}: absoluto
		\item \textcolor{red}{sqrt()}: Raíz cuadrada
		\item \textcolor{red}{nthroot (x,n)}: Enésima raíz real de x
		\item \textcolor{red}{sign()}: Regresa el signo del número
		\item \textcolor{red}{rem(x,y)}: Residuo de x/y
		\item \textcolor{red}{log()}: Calcula el logaritmo natural de x
		\item \textcolor{red}{log10()}: Calcula el logaritmo base 10 de x
	\end{itemize}

	\subsection{\textcolor{blue}{Funciones de Redondeo}}
	
	Las funciones de redondeo son:
	
	\begin{itemize}
		\item \textcolor{red}{round(b)}: Redondea al número entero más cercano
		\item \textcolor{red}{fix(b)}: Redondea al entero más cercano a cero
		\item \textcolor{red}{floor(b)}: Redondea al entero mas cercano hacia el infinito negativo
		\item \textcolor{red}{ceil(b)}:Redondea hacia el entero más cercano hacia el infinito positivo
	\end{itemize}

	\begin{figure}[h!]
		\centering
		\includegraphics[width = 100mm]{imagenes/math_discre}
		\caption{Funciones matemáticas discretas}
		\label{discretas}
	\end{figure}

	\section{Episodio 13: Funciones Trigonométricas}

	Los ángulos deben de estar en radianes
	
	\begin{equation}
	grados = radianes \left(\frac{180}{\pi}\right)
	\end{equation}
	
	\begin{equation}
	radianes = grados \left(\frac{\pi}{180}\right)
	\end{equation}
	
	Las funciones trigonométricas estandares son:
	\begin{multicols}{2}
	\begin{itemize}
		\item sin(x) con x en radianes
		\item cos(x) con x en radianes
		\item tan(x) con x en radianes
		\item csc(x) con x en radianes
		\item sec(x) con x en radianes
		\item cot(x) con x en radianes
		\item sind(x) con x en grados
		\item cosd(x) con x en grados
		\item tand(x) con x en grados
		\item cscd(x) con x en grados
		\item secd(x) con x en grados
		\item cotd(x) con x en grados
	\end{itemize}
	\end{multicols}

	\subsection{\textcolor{blue}{Funciones Trigonométricas Inversas}}

		Las funciones trigonométricas inversas son:
	\begin{multicols}{2}
		\begin{itemize}
			\item asin(x) con x en radianes
			\item acos(x) con x en radianes
			\item atan(x) con x en radianes
			\item acsc(x) con x en radianes
			\item asec(x) con x en radianes
			\item acot(x) con x en radianes
			\item asind(x) con x en grados
			\item acosd(x) con x en grados
			\item atand(x) con x en grados
			\item acscd(x) con x en grados
			\item asecd(x) con x en grados
			\item acotd(x) con x en grados
		\end{itemize}
	\end{multicols}

	\subsection{\textcolor{blue}{Funciones Trigonométricas Hiperbólicas}}

	\begin{multicols}{2}
	\begin{itemize}
		\item sinh(x) con x en radianes
		\item cosh(x) con x en radianes
		\item tanh(x) con x en radianes
		\item csch(x) con x en radianes
		\item sech(x) con x en radianes
		\item coth(x) con x en radianes
	\end{itemize}
\end{multicols}

	\section{Episodio 14: Análisis de Datos en MatLab}
	
	\begin{itemize}
		
	\item Para encontrar el \colorbox{orange}{máximo} de una matriz entrega el resultado del máximo por cada columna o un vector se utiliza la función \textcolor{green}{max()} y para guardar su posición se utiliza \textcolor{green}{[a,b] = max()} donde \colorbox{orange}{a} es el máximo y \colorbox{orange}{b} es su posición.

	\item Para encontrar el \colorbox{orange}{mínimo} se utiliza la funcinó \textcolor{green}{min()}
	
	\item Para \colorbox{orange}{trasponer} una matriz se utiliza el nombre de la matriz mas comilla; \textcolor{green}{max(y')}
	
	\item Para hacer una \colorbox{orange}{sumatoria} de los elementos de un vector o matriz se utiliza el comando \textcolor{green}{sum()}
	
	\item Para hacer una suma acumulada se tiene que \colorbox{orange}{cumsum()} que acumula la suma por filas
	
	\item De la misma manera funciona el \colorbox{orange}{productorio} con los comandos \textcolor{green}{prod()} y \textcolor{green}{cumprod()}
	
	\item Para ordenar los datos de forma ascendente o descendente se utiliza el comando \textcolor{green}{sort()} y de manera descendente \textcolor{green}{sort(x, 'descend')}
	
	\item Para ordenar una columna determinada se utiliza el comando \textcolor{green}{sortrows(y,n)}
	
	\item El comando \textcolor{green}{size} determina el tamaño de la matriz y \textcolor{green}{length} arroja el valor mayor del tamaño de la matriz
	
	\end{itemize}
	\section{Episodio 15: Números Complejos}
	
	MatLab incluye varias funciones principales con números complejos.
	Se representan como:
	\begin{itemize}
		\item a = 12 + 7i ó a = 12 + 7j o también se puede utilizar el comando \textcolor{green}{complex(a,i)}; para que lo anterior funciones, las variables i y j no deben de estar siendo empleadas. Lo anterior también puede ser usado con vectores.

		\item Se puede llamar la parte real de un número complejo y la parte imaginaria con los comandos \textcolor{green}{real()} y \textcolor{green}{imag()}
	
		\item El comando \textcolor{green}{isreal()} se emplea para conocer si el número es real o no con indicadores lógicos.
		El conjugado de un número complejo se obtiene con \textcolor{green}{conj()}
	
		\item  Para escribir un número complejo en coordenadas polares se utiliza los comandos \textcolor{green}{abs()} para encontrar el radio y el comando \textcolor{green}{angle} para determinar el ángulo	
	\end{itemize}
	
	\section{Episodio 16: Graficar Vectores en 2D}
	
	\begin{itemize}
	\item Se definen los datos de vectores  $x$ y $y$. Luego se utiliza el comando \textcolor{green}{plot}. Para agregar un título a la gráfica se utiliza el comando \textcolor{green}{title('')}. Para nombrar los ejes se pone \textcolor{green}{xlable('')} y \textcolor{green}{ylabel('')}. Para poner una grilla se utiliza \textcolor{green}{grid on} y el comando \textcolor{green}{leyend} pone nombres en la gráfica.
	
	\item El comando \textcolor{green}{hold} y luego el comando \textcolor{green}{plot} para sobreponer las gráficas. Para agregar las leyendas a las gráficas se utiliza el comando \textcolor{green}{legend('Primera función', 'Segunda función')}
	
	\subsection{Creación de Gráficas Múltiples}
	
	\item Con \textcolor{green}{figure ()} se crean las figuras en ventanas diferentes
	
	\subsection{Subplot}
	
	\item Para crear varias figuras en una misma ventana se utiliza \textcolor{green}{figure} y \textcolor{green}{subplot(2,1,1)} donde el primer número es el número de filas, el segundo el número de columnas y el tercero el axe actual
	
	\subsection{Línea, Color y Estilo de Marca}
	
	Los diferentes estilos de gráficas son:
	
	\begin{figure}[h!]
		\centering
		\includegraphics[width = 100mm]{imagenes/atributos-MATLAB}
		\label{estilos}
		\caption{Estilos y colores en gráficas}
	\end{figure}

	\item Para ello se pone \textcolor{green}{plot(a,b,'atributos')}
	
	\item Para cambiar el tamaño de la línea se utiliza el comando \textcolor{green}{linewigth} y para el tamaño de la fuente \textcolor{green}{FontSize}. Para cambiar la fuente de los Axes se usa el comando \textcolor{green}{gca}.
	
	\item \textcolor{green}{plot(a,b,'atributos','linewitdh',6)}
	
	\item \textcolor{green}{title('Función de Seno','FontSize',15)}
	
	\item \textcolor{green}{set(gca,'fontsize',14)}
	\subsection{Escalamiento de Ejes}
	
	\item Se utilizan los comandos \textcolor{blue}{axis([xmin xmax ymin ymax])}
	
	\item Para crear anotaciones dentro de la gráfica se utiliza el comando \textcolor{blue}{text(Posición en $x$, Posición en $y$, 'Leyenda')}
	
	\item Para controlar más sobre los parámetros de plot, utilizar el comando \textcolor{blue}{help plot}
	
	\end{itemize}
	
\end{document}